\documentclass[12pt]{article}

% Layout.
\usepackage[top=1.2in, bottom=0.75in, left=1in, right=1in, headheight=1.0in, headsep=0pt]{geometry}

% Fonts.
\usepackage{mathptmx}
\usepackage[scaled=0.86]{helvet}
\renewcommand{\emph}[1]{\textsf{\textbf{#1}}}

% TiKZ.
\usepackage{tikz, pgfplots}
\usetikzlibrary{calc}
\pgfplotsset{my style/.append style={axis x line=middle, axis y line=middle, xlabel={$x$}, ylabel={$y$}}}
\pgfplotsset{compat=1.16}

% Misc packages.
\usepackage{amsmath,amssymb,latexsym}
\usepackage{graphicx}
\usepackage{array}
\usepackage{xcolor}
\usepackage{multicol}

% Commands to set various header/footer components.
\makeatletter
\def\doctitle#1{\gdef\@doctitle{#1}}
\doctitle{Use {\tt\textbackslash doctitle\{MY LABEL\}}.}
\def\docdate#1{\gdef\@docdate{#1}}
\docdate{Use {\tt\textbackslash docdate\{MY DATE\}}.}
\def\doccourse#1{\gdef\@doccourse{#1}}
\let\@doccourse\@empty
\def\docscoring#1{\gdef\@docscoring{#1}}
\let\@docscoring\@empty
\def\docversion#1{\gdef\@docversion{#1}}
\let\@docversion\@empty
\makeatother

% Headers and footers layout.
\makeatletter
\usepackage{fancyhdr}
\pagestyle{fancy}
\fancyhf{} % Clears all headers/footers.
\lhead{\emph{\@doctitle\hfill\@docdate} \medskip
\ifnum \value{page} > 1\relax\else\\
\emph{Name: \rule{3.5in}{1pt}\hfill \@docscoring}
\fi}

\rfoot{\emph{\@docversion}}
\lfoot{\emph{\@doccourse}}
\cfoot{\emph{\thepage}}
\renewcommand{\headrulewidth}{0pt}%
\makeatother

% Paragraph spacing
\parindent 0pt
\parskip 6pt plus 1pt

% A problem is a section-like command. Use \problem{5} for a problem worth 5 points.
\newcounter{probcount}
\newcounter{subprobcount}
\setcounter{probcount}{0}

\newcommand{\problem}[1]{%
\par
\addvspace{4pt}%
\setcounter{subprobcount}{0}%
\stepcounter{probcount}%
\makebox[0pt][r]{\emph{\arabic{probcount}.}\hskip1ex}\emph{[#1 points]}\hskip1ex}

\newcommand{\thesubproblem}{\emph{\alph{subprobcount}.}}

% like \problem but with name
\newcommand{\nameprob}[2]{%
\par
\addvspace{4pt}%
\setcounter{subprobcount}{0}%
\stepcounter{probcount}%
\makebox[0pt][r]{\emph{#1.}\hskip1ex}\emph{[#2 points]}\hskip1ex}

% Subproblems are an enumerate-like environment with a consistent
% numbering scheme.  Use \begin{subproblems}\item...\item...\end{subproblems}
\newenvironment{subproblems}{%
\begin{enumerate}%
\setcounter{enumi}{\value{subprobcount}}%
\renewcommand{\theenumi}{\emph{\alph{enumi}}}}%
{\setcounter{subprobcount}{\value{enumi}}\end{enumerate}}

% Blanks for answers in normal and math mode.
\newcommand{\blank}[1]{\rule{#1}{0.75pt}}
\newcommand{\mblank}[1]{\underline{\hspace{#1}}}
\def\emptybox(#1,#2){\framebox{\parbox[c][#2]{#1}{\rule{0pt}{0pt}}}}

% Misc.
\renewcommand{\d}{\displaystyle}
\newcommand{\ds}{\displaystyle}


\doctitle{Math 252: Quiz 6}
\docdate{20 October, 2022}
\doccourse{}
\docversion{}
\docscoring{\fbox{{\LARGE \strut}\hspace{0.8in} / 25}}

\begin{document}
30 minutes maximum.  No aids (book, calculator, etc.) are permitted.  Show all work and use proper notation for full credit.  Answers should be in reasonably-simplified form.  25 points possible.

\problem{9}  Compute and simplify the improper integrals, or show that they diverge.  Use correct limit notation.

\begin{subproblems}
\item $\displaystyle \int_0^\infty \frac{1}{4+x^2}\,dx = $
\vfill
\item $\displaystyle \int_0^\infty \sin x \,dx = $
\vfill
\item $\displaystyle \int_0^1 \frac{1}{\sqrt[4]{x}} \,dx = $
\vfill
\end{subproblems}

\clearpage\newpage
\problem{4}  Verify that {\large $y = e^{x^2/2}$} solves the differential equation {\large $y'=xy$}.  %#12 in 4.1
\vspace{3.5in}

\problem{4}  Find the general solution to the differential equation {\large $y'=\ln x + \tan x$}.  %#29 in 4.1
\vfill

\clearpage\newpage
\problem{4}  Find the particular solution of the differential equation {\large $\displaystyle y'=\frac{y}{x^2}$} that passes through $\displaystyle \left(1,\frac{2}{e}\right)$, given that {\large $\displaystyle y=C e^{-1/x}$} is the general solution.  %#23 in 4.1
\vfill

\problem{4}  Find the area of the region in the first quadrant between the curve $y=e^{-3x}$ and the $x$-axis.  %#394 in 3.7, essentially
\vfill

\newpage
\nameprob{EC}{1} (\emph{Extra Credit})  Suppose the curve $y=e^{-x}$, on the interval $[0,+\infty)$, is rotated around the $x$-axis.  Set up an integral for the surface area and give a convincing argument that this integral is finite.
\vfill

\noindent \hrule

\bigskip
\centerline{\footnotesize \textsc{blank space}}
\vfill
\end{document}