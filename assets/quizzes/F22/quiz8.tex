\documentclass[12pt]{article}

% Layout.
\usepackage[top=1.2in, bottom=0.75in, left=1in, right=1in, headheight=1.0in, headsep=0pt]{geometry}

% Fonts.
\usepackage{mathptmx}
\usepackage[scaled=0.86]{helvet}
\renewcommand{\emph}[1]{\textsf{\textbf{#1}}}

% TiKZ.
\usepackage{tikz, pgfplots}
\usetikzlibrary{calc}
\pgfplotsset{my style/.append style={axis x line=middle, axis y line=middle, xlabel={$x$}, ylabel={$y$}}}
\pgfplotsset{compat=1.16}

% Misc packages.
\usepackage{amsmath,amssymb,latexsym}
\usepackage{graphicx}
\usepackage{array}
\usepackage{xcolor}
\usepackage{multicol}

% Commands to set various header/footer components.
\makeatletter
\def\doctitle#1{\gdef\@doctitle{#1}}
\doctitle{Use {\tt\textbackslash doctitle\{MY LABEL\}}.}
\def\docdate#1{\gdef\@docdate{#1}}
\docdate{Use {\tt\textbackslash docdate\{MY DATE\}}.}
\def\doccourse#1{\gdef\@doccourse{#1}}
\let\@doccourse\@empty
\def\docscoring#1{\gdef\@docscoring{#1}}
\let\@docscoring\@empty
\def\docversion#1{\gdef\@docversion{#1}}
\let\@docversion\@empty
\makeatother

% Headers and footers layout.
\makeatletter
\usepackage{fancyhdr}
\pagestyle{fancy}
\fancyhf{} % Clears all headers/footers.
\lhead{\emph{\@doctitle\hfill\@docdate} \medskip
\ifnum \value{page} > 1\relax\else\\
\emph{Name: \rule{3.5in}{1pt}\hfill \@docscoring}
\fi}

\rfoot{\emph{\@docversion}}
\lfoot{\emph{\@doccourse}}
\cfoot{\emph{\thepage}}
\renewcommand{\headrulewidth}{0pt}%
\makeatother

% Paragraph spacing
\parindent 0pt
\parskip 6pt plus 1pt

% A problem is a section-like command. Use \problem{5} for a problem worth 5 points.
\newcounter{probcount}
\newcounter{subprobcount}
\setcounter{probcount}{0}

\newcommand{\problem}[1]{%
\par
\addvspace{4pt}%
\setcounter{subprobcount}{0}%
\stepcounter{probcount}%
\makebox[0pt][r]{\emph{\arabic{probcount}.}\hskip1ex}\emph{[#1 points]}\hskip1ex}

\newcommand{\thesubproblem}{\emph{\alph{subprobcount}.}}

% like \problem but with name
\newcommand{\nameprob}[2]{%
\par
\addvspace{4pt}%
\setcounter{subprobcount}{0}%
\stepcounter{probcount}%
\makebox[0pt][r]{\emph{#1.}\hskip1ex}\emph{[#2 points]}\hskip1ex}

% Subproblems are an enumerate-like environment with a consistent
% numbering scheme.  Use \begin{subproblems}\item...\item...\end{subproblems}
\newenvironment{subproblems}{%
\begin{enumerate}%
\setcounter{enumi}{\value{subprobcount}}%
\renewcommand{\theenumi}{\emph{\alph{enumi}}}}%
{\setcounter{subprobcount}{\value{enumi}}\end{enumerate}}

% Blanks for answers in normal and math mode.
\newcommand{\blank}[1]{\rule{#1}{0.75pt}}
\newcommand{\mblank}[1]{\underline{\hspace{#1}}}
\def\emptybox(#1,#2){\framebox{\parbox[c][#2]{#1}{\rule{0pt}{0pt}}}}

% Misc.
\renewcommand{\d}{\displaystyle}
\newcommand{\ds}{\displaystyle}


\doctitle{Math 252: Quiz 8}
\docdate{3 November, 2022}
\doccourse{}
\docversion{}
\docscoring{\fbox{{\LARGE \strut}\hspace{0.8in} / 25}}

\begin{document}
30 minutes maximum.  No aids (book, calculator, etc.) are permitted.  Show all work and use proper notation for full credit.  Answers should be in reasonably-simplified form.  25 points possible.

\problem{9}  Use either the \emph{comparison test} or the \emph{limit comparison test} to determine whether the series converge or diverge.  Please show enough work to demonstrate how the test is applied.

\begin{subproblems}
\item $\displaystyle \sum_{n=1}^\infty \frac{1}{2n+1}$
\vfill

\item $\displaystyle \sum_{n=2}^\infty \frac{1}{n^2 + \ln n}$
\vfill

\item $\displaystyle \sum_{n=1}^\infty \frac{\sin^2 n}{2^n}$
\vfill
\end{subproblems}


\clearpage\newpage
\problem{4}  Apply the \emph{integral test} to show that $\displaystyle \sum_{n=1}^\infty \frac{1}{n^p}$ converges for $p>1$.
\vfill

\problem{3}  Sketch a partial sum $S_N$ of the harmonic series, as the total area of rectangles of width one.  (\emph{Please label axes appropriately.})
\vfill


\clearpage\newpage
\problem{9}  Use an appropriate test to determine whether the series converge or diverge.  Please state what test you are using, and show enough work to demonstrate how the test is applied.

\begin{subproblems}
\item $\displaystyle \sum_{n=2}^\infty \frac{(n-2)!}{n!}$
\vfill

\item $\displaystyle \sum_{n=2}^\infty \frac{1}{n (\ln n)^2}$
\vfill

\item $\displaystyle \sum_{n=1}^\infty \frac{n^2}{\sqrt{n^3 + 1}}$
\vfill
\end{subproblems}

\newpage
\nameprob{EC}{1} (\emph{Extra Credit})  My computer says that the 100th partial sum of the series $\displaystyle S=\sum_{n=1}^\infty \frac{1}{n^3}$ is $S_{100}=1.2020074$.  How accurate is this?  Use an integral to estimate the size of the remainder $R_{100} = S - S_{100}$.
\vfill

\noindent \hrule

\bigskip
\centerline{\footnotesize \textsc{blank space}}
\vfill
\end{document}