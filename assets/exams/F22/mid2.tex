\documentclass[11pt]{amsart}
%\pagestyle{empty} 
\setlength{\topmargin}{-0.5in} % usually -0.25in
\addtolength{\textheight}{1.2in} % usually 1.25in
\addtolength{\oddsidemargin}{-1.0in}
\addtolength{\evensidemargin}{-1.0in}
\addtolength{\textwidth}{1.9in} %\setlength{\parindent}{0pt}

\newcommand{\normalspacing}{\renewcommand{\baselinestretch}{1.1}\tiny\normalsize}
\normalspacing

% macros
\usepackage{amssymb,xspace,alltt,verbatim}
\usepackage[final]{graphicx}
\usepackage[pdftex,colorlinks=true]{hyperref}
\usepackage{fancyvrb}
\usepackage{tikz}

\newtheorem*{lem*}{Lemma}

\newcommand{\bb}{\mathbf{b}}
\newcommand{\bc}{\mathbf{c}}
\newcommand{\bs}{\mathbf{s}}
\newcommand{\bu}{\mathbf{u}}
\newcommand{\bv}{\mathbf{v}}
\newcommand{\bw}{\mathbf{w}}
\newcommand{\bx}{\mathbf{x}}
\newcommand{\by}{\mathbf{y}}

\newcommand{\bbf}{\mathbf{f}}

\newcommand{\CC}{{\mathbb{C}}}
\newcommand{\RR}{{\mathbb{R}}}
\newcommand{\eps}{\epsilon}
\newcommand{\ZZ}{{\mathbb{Z}}}
\newcommand{\ZZn}{{\mathbb{Z}}_n}
\newcommand{\NN}{{\mathbb{N}}}
\newcommand{\ip}[2]{\mathrm{\left<#1,#2\right>}}

\renewcommand{\Re}{\operatorname{Re}}
\renewcommand{\Im}{\operatorname{Im}}

\newcommand{\Log}{\operatorname{Log}}

\newcommand{\grad}{\nabla}

\newcommand{\ds}{\displaystyle}

\newcommand{\Matlab}{\textsc{Matlab}\xspace}
\newcommand{\Octave}{\textsc{Octave}\xspace}
\newcommand{\pylab}{\textsc{pylab}\xspace}

\newcommand{\prob}[1]{\bigskip\noindent\textbf{#1.} }
\newcommand{\pts}[1]{(\emph{#1 pts})}

\newcommand{\probpts}[2]{\prob{#1} \pts{#2} \quad}
\newcommand{\ppartpts}[2]{\textbf{(#1)} \pts{#2} \quad}
\newcommand{\epartpts}[2]{\medskip\noindent \textbf{(#1)} \pts{#2} \quad}


\begin{document}
\hfill \Large Name:\underline{\phantom{REALLY REALLY really really long long long name}}
\medskip

\scriptsize \noindent Math 252 Calculus 2 (Bueler) \hfill 17 November 2022
\medskip

\LARGE\centerline{\textbf{Midterm Exam 2}}

\smallskip
\begin{quote}
\large
\textbf{No book, electronics, calculator, or internet access.  Only ``Summary of Convergence Tests'' notes allowed.  100 points possible.  70 minutes.}
\end{quote}

\normalsize
\medskip

\thispagestyle{empty}

\probpts{1}{5}  Verify that $y=e^{2x^2}$ is a solution to the differential equation $\ds y'-4xy=0$.

\vfill

\prob{2}  Compute and simplify the improper integrals, or show they diverge.  Use correct limit notation.

\epartpts{a}{5} $\ds \int_0^1 \frac{dx}{x^3} =$
\vfill

\epartpts{b}{5} $\ds \int_1^\infty 2xe^{-x^2}\, dx = $
\vfill


\clearpage\newpage
\prob{3}  Do the following series converge or diverge?  Show your work, including naming any test you use.

\epartpts{a}{5}  $\ds \sum_{n=1}^\infty \frac{n+1}{n^2}$
\vfill

\epartpts{b}{5}  $\ds \sum_{n=1}^\infty \frac{2n+1}{5n-1}$
\vfill

\epartpts{c}{5}  $\ds \sum_{n=1}^\infty \frac{(-1)^{n+1}}{\sqrt{2n}}$
\vfill

\clearpage\newpage
\epartpts{d}{5}  $\ds \sum_{n=0}^\infty \frac{2^n n!}{(n+2)!}$
\vfill

\epartpts{e}{5}  $\ds \sum_{n=0}^\infty \left(\frac{n+3}{2n-1}\right)^n$
\vfill


\clearpage\newpage
\probpts{4}{5}  Does the following series converge or diverge?  Show your work, including naming any test you use.  (\emph{Hint.}  Previous problem?  Or another test?) \large
  $$\sum_{n=1}^\infty \frac{2n}{e^{(n^2)}} \hspace{4.5in}$$
\normalsize
\vfill

\probpts{5}{5}  Compute and simplify the value of the infinite series $\ds \sum_{n=1}^\infty \left(\frac{2}{5}\right)^{n+1}$.
\vfill


\clearpage\newpage
\prob{6}  Consider the infinite series $\ds 1-\frac{1}{4}+\frac{1}{9}-\frac{1}{16}+\frac{1}{25} - \frac{1}{36} +\dots$

\epartpts{a}{5}  Write the series using sigma ($\sum$) notation.
\vfill

\epartpts{b}{5}  Compute and simplify $S_3$, the partial sum of the first three terms.
\vfill

\newcommand{\threeopts}{{\small \hspace{-6mm} $\begin{matrix} \text{\textsc{converges}} \\ \text{\textsc{absolutely}} \end{matrix}$ \qquad\qquad $\begin{matrix} \text{\textsc{converges}} \\ \text{\textsc{conditionally}} \end{matrix}$ \qquad\qquad \textsc{diverges}} \bigskip}

\epartpts{c}{5} Does this series converge absolutely, conditionally, or neither (diverge)?  Show your work, identify any test(s) used, and circle one answer. 
\vfill

\threeopts


\clearpage\newpage
\prob{7}  Use the well known geometric series $\ds \frac{1}{1-r}=\sum_{n=0}^\infty r^n$ to find power series representations for the following functions.  Show your work.

\epartpts{a}{5}  $\ds \frac{1}{1+x^3}$

\vspace{1.4in}

\noindent $\ds \frac{1}{1+x^3}=$
\vspace{0.3in}

\epartpts{b}{7}  $\ds \ln(1+x)$

\vfill

\noindent $\ds \ln(1+x) =$
\vspace{0.3in}

\probpts{8}{7} If $\ds f(x)=\sum_{n=1}^\infty \frac{x^n}{\sqrt{n}}$, find a power series representation for $f'(5x)$

\vfill

\noindent $f'(5x)=$


\clearpage\newpage
\prob{9}  Find the interval of convergence of the following power series.

\epartpts{a}{8}  $\ds \sum_{n=1}^\infty \frac{2^nx^n}{n!}$
\vfill

\epartpts{b}{8}  $\ds \sum_{n=1}^\infty \frac{(x-1)^n}{n\, 3^n}$
\vfill


\clearpage\newpage
\probpts{Extra Credit}{3}  The function $f(x)=\arctan(x)$ can be represented by the power series
    $$\arctan(x) = \sum_{n=0}^\infty \frac{(-1)^n x^{2n+1}}{2n+1}.$$
Suppose I choose $x=1/\sqrt{3}$ and compute the partial sum $\ds S_{20} = \sum_{n=0}^{20} \frac{(-1)^n (1/\sqrt{3})^{2n+1}}{2n+1}$ as an approximation to $\ds \arctan(1/\sqrt{3}) = \frac{\pi}{6}$.

\medskip
\textbf{How accurate is this approximation?}  Use a known fact about remainders of alternating series.
\vspace{2.0in}

$\ds \left|S_{20} - \frac{\pi}{6}\right| \le \boxed{\phantom{\begin{matrix} lakjsd alsfjd \\ sdfa  asdf as ads fasdd \\ sdaf \end{matrix}}}$

\bigskip
\noindent \hrule
\begin{center}
\small
\bigskip
\textsc{blank space}
\end{center}
\vfill

\end{document}
