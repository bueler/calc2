\documentclass[12pt]{amsart}
%prepared in AMSLaTeX, under LaTeX2e
\addtolength{\oddsidemargin}{-.65in}
\addtolength{\evensidemargin}{-.65in}
\addtolength{\topmargin}{-.4in}
\addtolength{\textwidth}{1.3in}
\addtolength{\textheight}{.75in}

\renewcommand{\baselinestretch}{1.05}

\usepackage{xspace}
\usepackage{verbatim,fancyvrb}
\usepackage{palatino}
\usepackage[final]{graphicx}

\newtheorem*{thm}{Theorem}
\newtheorem*{defn}{Definition}
\newtheorem*{example}{Example}
\newtheorem*{problem}{Problem}
\newtheorem*{remark}{Remark}

\newcommand{\mfile}[1]{
\begin{quote}
\bigskip
\VerbatimInput[frame=single,framesep=3mm,label=\fbox{\normalsize \textsl{\,#1\,}},fontfamily=courier,fontsize=\footnotesize]{../matlab/#1}
\medskip
\end{quote}
}

%\DefineVerbatimEnvironment{mVerb}{Verbatim}{numbersep=2mm,frame=lines,framerule=0.1mm,framesep=2mm,xleftmargin=4mm,fontsize=\footnotesize}
\DefineVerbatimEnvironment{mVerb}{Verbatim}{numbersep=2mm,xleftmargin=4mm,fontsize=\footnotesize}

% macros
\usepackage{amssymb}

\newcommand{\br}{\mathbf{r}}
\newcommand{\bv}{\mathbf{v}}
\newcommand{\bx}{\mathbf{x}}
\newcommand{\by}{\mathbf{y}}

\newcommand{\CC}{\mathbb{C}}
\newcommand{\RR}{\mathbb{R}}
\newcommand{\ZZ}{\mathbb{Z}}

\newcommand{\eps}{\epsilon}
\newcommand{\grad}{\nabla}
\newcommand{\lam}{\lambda}
\newcommand{\lap}{\triangle}

\newcommand{\prob}[1]{\bigskip\noindent\textbf{#1.}\quad }

\newcommand{\pts}[1]{(\emph{#1 pts}) }
\newcommand{\epart}[1]{\medskip\noindent\textbf{(#1)}\quad }
\newcommand{\ppart}[1]{\,\textbf{(#1)}\quad }

\newcommand{\ds}{\displaystyle}

\begin{document}
\scriptsize \noindent Math 252 Calculus II (Bueler) \hfill Due \underline{Wednesday 2 November 2022 at 11:59pm}
\normalsize

\Large
\bigskip
\centerline{\textbf{Homework 5.4: Replacement problems}}
\medskip
\normalsize

\thispagestyle{empty}

\bigskip

The Exercises in this section are not well-aligned to the text!  Here are replacements.

\bigskip
\noindent Use the comparison test to determine whether the series converge or diverge.

\newcommand{\pr}[1]{\bigskip \textbf{#1.} \,\,}

\pr{1} $\ds \sum_{n=1}^\infty a_n$ where $\ds a_n = \frac{2}{n(n+1)}$

\pr{2} $\ds \sum_{n=1}^\infty \frac{1}{2(n+1)}$

\pr{3} $\ds \sum_{n=1}^\infty \frac{1}{2n - 1}$

\pr{4} $\ds \sum_{n=2}^\infty \frac{1}{(n\ln n)^2}$

\pr{5} $\ds \sum_{n=1}^\infty \frac{n!}{(n+2)!}$

\pr{6} $\ds \sum_{n=1}^\infty \frac{1}{n!}$

\pr{7} $\ds \sum_{n=1}^\infty \frac{\sin^2 n}{n^2}$

\bigskip
\noindent Use the limit comparison test to determine whether the series converge or diverge.

\pr{8} $\ds \sum_{n=0}^\infty \frac{1}{n+3}$

\pr{9} $\ds \sum_{n=2}^\infty \frac{1}{n^2 - \sqrt{n}}$

\pr{10} $\ds \sum_{n=1}^\infty \frac{3^n}{5^n + 4^n}$

\pr{11} $\ds \sum_{n=1}^\infty \frac{\ln n}{n^3}$

\pr{12} $\ds \sum_{n=1}^\infty \frac{1}{4^n - 3^n}$

\bigskip
\pr{13} Does $\ds \sum_{n=2}^\infty \frac{1}{(\ln n)^p}$ converge if $p$ is large enough?  If so, for which $p$?

\pr{14} For which $p$ does the series $\ds \sum_{n=1}^\infty \frac{2^{pn}}{3^n}$ converge?  Why?

\pr{15} For which $p>0$ does the series $\ds \sum_{n=1}^\infty \frac{n^p}{2^n}$ converge?  Why?

\pr{16} \begin{minipage}[t]{5.5in} Suppose that $a_n>0$ and that $\ds \sum_{n=1}^\infty a_n$ converges.  Suppose also that $b_n$ is an arbitrary sequence of zeros and ones. Does $\ds \sum_{n=1}^\infty a_n b_n$ necessarily converge?  Explain using a convergence test. \end{minipage}

\pr{17} \begin{minipage}[t]{5.5in} Show that if $a_n \ge 0$ and $\ds \sum_{n=1}^\infty a_n^2$ converges, then $\ds \sum_{n=1}^\infty \sin^2(a_n)$ converges.  (\emph{Hint. Use a convergence test, and the fact that $|\sin \theta| \le |\theta|$.})
\end{minipage}

\pr{18} \begin{minipage}[t]{5.5in} Let $d_n$ be an infinite sequence of digits, meaning $d_n$ takes values in $\{0,1,\dots,9\}$.  What is the largest possible value of $\ds x = \sum_{n=1}^\infty \frac{d_n}{10^n}$ that converges?
\end{minipage}

\end{document}
