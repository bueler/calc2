\documentclass[12pt]{amsart}
%prepared in AMSLaTeX, under LaTeX2e
\addtolength{\oddsidemargin}{-.65in} 
\addtolength{\evensidemargin}{-.65in}
\addtolength{\topmargin}{-.4in}
\addtolength{\textwidth}{1.3in}
\addtolength{\textheight}{.75in}

\renewcommand{\baselinestretch}{1.05}

\usepackage{xspace}
\usepackage{verbatim,fancyvrb}
\usepackage{palatino}
\usepackage[final]{graphicx}

\newtheorem*{thm}{Theorem}
\newtheorem*{defn}{Definition}
\newtheorem*{example}{Example}
\newtheorem*{problem}{Problem}
\newtheorem*{remark}{Remark}

\newcommand{\mfile}[1]{
\begin{quote}
\bigskip
\VerbatimInput[frame=single,framesep=3mm,label=\fbox{\normalsize \textsl{\,#1\,}},fontfamily=courier,fontsize=\footnotesize]{../matlab/#1}
\medskip
\end{quote}
}

%\DefineVerbatimEnvironment{mVerb}{Verbatim}{numbersep=2mm,frame=lines,framerule=0.1mm,framesep=2mm,xleftmargin=4mm,fontsize=\footnotesize}
\DefineVerbatimEnvironment{mVerb}{Verbatim}{numbersep=2mm,xleftmargin=4mm,fontsize=\footnotesize}

% macros
\usepackage{amssymb}

\newcommand{\br}{\mathbf{r}}
\newcommand{\bv}{\mathbf{v}}
\newcommand{\bx}{\mathbf{x}}
\newcommand{\by}{\mathbf{y}}

\newcommand{\CC}{\mathbb{C}}
\newcommand{\RR}{\mathbb{R}}
\newcommand{\ZZ}{\mathbb{Z}}

\newcommand{\eps}{\epsilon}
\newcommand{\grad}{\nabla}
\newcommand{\lam}{\lambda}
\newcommand{\lap}{\triangle}

\newcommand{\ip}[2]{\ensuremath{\left<#1,#2\right>}}

\newcommand{\image}{\operatorname{im}}
\newcommand{\onull}{\operatorname{null}}
\newcommand{\rank}{\operatorname{rank}}
\newcommand{\range}{\operatorname{range}}
\newcommand{\trace}{\operatorname{tr}}

\newcommand{\prob}[1]{\bigskip\noindent\textbf{#1.}\quad }
\newcommand{\exer}[1]{\prob{Exercise #1}}

\newcommand{\pts}[1]{(\emph{#1 pts}) }
\newcommand{\epart}[1]{\medskip\noindent\textbf{(#1)}\quad }
\newcommand{\ppart}[1]{\,\textbf{(#1)}\quad }

\newcommand{\Matlab}{\textsc{Matlab}\xspace}
\newcommand{\Octave}{\textsc{Octave}\xspace}
\newcommand{\Python}{\textsc{Python}\xspace}
\newcommand{\Julia}{\textsc{Julia}\xspace}


\begin{document}
\scriptsize \noindent Math 252 Calculus II (Bueler) \hfill \emph{12/30/21}
\normalsize

\Large
\bigskip
\centerline{\textbf{Special instructions for \S 3.6 Homework}}
\medskip
\normalsize

\thispagestyle{empty}

\bigskip

I am not a big fan of how the Exercises in this section are written!  Here are clarifications and special instructions for a few of the Exercises.

\bigskip
\noindent \textbf{Exercise 306.}  ``error of approximation'' simply means absolute error.

\medskip
\noindent \textbf{Exercise 310 and 311.}  A very accurate numerical integration gives $0.657669856$ for this integral.  For each Exercise, compute the absolute error.  Which method is more accurate?

\medskip
\noindent \textbf{Exercise 314 and 315.}  A very accurate numerical integration gives $1.55008431$ for this integral.  For each Exercise, compute the absolute error.  Which method is more accurate?

\medskip
\noindent \textbf{Exercise 322.}  Use 6 subdivisions.  (\emph{Instead of 16, which is tedious.})

\end{document}
